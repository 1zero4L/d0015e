% skapa en "article" med 12 punkters text och små marginaler
\documentclass[12pt,a4wide]{article} 
%
\usepackage[utf8]{inputenc}  % teckenkodningen är UTF-8
\usepackage[swedish]{babel}  % detta dokument skrivs på svenska
%
\usepackage{amsmath}         % tillför extra stöd för matematik
\usepackage{amsthm}          % tillför extra stöd för matematik
\usepackage{enumerate}       % ger flexibel listnumrering, om man vill
\usepackage[final]{graphicx} % för att kunna inkludera bilder
%
\newtheorem{exempel}{Exempel} % för att kunna visa exempel i texten
%
% Nedanstående 14 rader kan skippas; de definierar de nya
% omgivningarna "uppgift" och "avklarad"
%
\usepackage{framed}
\usepackage{wasysym}  
\newtheoremstyle{uppgiftsstil}{0pt}{0pt}{}{}{\bfseries}{.}{.5em}{}
\theoremstyle{uppgiftsstil} 
\newcommand{\ovningstext}{Övningsuppgift}
\newtheorem{ovning}{\ovningstext} 
\newenvironment{uppgift}{\begin{framed}\begin{ovning}}%
                        {\end{ovning}\end{framed}}
\newtheoremstyle{avklaradstil}{0pt}{0pt}{}{}{}{!~\smiley}{.5em}{}
\theoremstyle{avklaradstil} 
\newcommand{\avklaradtext}{Avklarad övningsuppgift}
\newtheorem{klar}[ovning]{\avklaradtext}
\newenvironment{avklarad}{\begin{framed}\begin{klar}}%
                         {\end{klar}\end{framed}}
       
\title{Om att chansa på traditionella kryssfrågeprov lönar sig}  % titeln på hela rapporten

% författaren med kontaktinformation
\author{Ion Lund\thanks{Email:~\texttt{ionlun-5@student.ltu.se}} \\
        Luleå tekniska universitet \\  % två bakåtsnedstreck betyder "ny rad"
        97187 Luleå}       

\begin{document}         % börja att typsätta här

\maketitle               % typsätt titeln 

\begin{abstract}         % sammanfattningen, som hamnar efter titeln
  I denna rapport redovisar vi hur vi har löst två uppgifter
  som handlar om hur många poäng man kan förvänta sig att få om man
  chansar sig genom ett prov med kryssfrågor.
\end{abstract}

\section{Introduktion} % första avsnittet i rapporten
\label{sec:intro} % en etikett på första avsnittet

\begin{avklarad}
  Ändra titeln så att den istället lyder: ''Att chansa på
  traditionella kryssfrågeprov lönar sig''. 

  Generera ett nytt dokument baserat på \LaTeX-manuset och
  kolla att det blir rätt. Experimentera gärna med
  olika varianter innan du går vidare! 

  När du är klar med uppgiften så byt namn på själva
  uppgiftsomgivningen från \texttt{uppgift} till
  \texttt{avklarad}. Ändra alltså \verb|\begin{uppgift}| och
  \verb|\end{uppgift}| till \verb|\begin{avklarad}| respektive
  \verb|\end{avklarad}|. Detta ändrar texten överst till vänster i
  rutan så att du enkelt kan se hur långt du kommit om du t ex behöver
  ta en paus.   

  Ovanstående gäller för alla övningsuppgifter nedan trots att det
  inte upprepas. 
\end{avklarad}
%
\begin{avklarad}
  Ändra så att du står som författare. Byt även ut mailadressen
  till din egen studentmail. 
\end{avklarad}
% nästa rad, som är tom, betyder "nytt stycke"

I denna rapport, som är en del av kursmaterialet i kursen D0015E \label{d0015e} datateknik och ingenjörsvetenskap vid Luleå tekniska universitet,
redovisar vi hur vi har löst följande två uppgifter som berör chansningar på kryssfrågeprov:
%
\begin{avklarad}
  Skriv om ovanstående (långa) mening till två meningar med samma
  betydelse, där den första meningen förklarar att rapporten är en del
  av kursmaterialet och den andra säger att rapporten berör
  chansningar på kryssfrågeprov.  Se till att etiketten
  (\texttt{d0010e}) fortfarande följer direkt efter kurskoden. 
\end{avklarad}
%
\begin{enumerate}            % en lista, en uppräkning av "item":s
  \item Vad är det förväntade antalet poäng man får på en enskild
    fråga om man chansar utan att ens titta på
    svarsalternativen? \label{item:question}  % etikett  
%
  \item Hur många poäng kan man totalt förvänta sig att få på ett prov
    om man chansar på alla frågor? \label{item:test} % ännu en etikett 
\end{enumerate}              % slut på listan
%
Uppgifterna behandlas i var sitt avsnitt nedan. 
\begin{uppgift}
  Ange några anledning till att denna rapport skrivits. Använd en
  \texttt{enumerate}-omgivning och räkna upp följande anledningar:
  Lära sig mer om chansningar och lära sig \LaTeX.  
\end{uppgift}

\section{Definitioner och inledande beteckningar} 

För att resonera om kryssfrågor och prov inför vi några definitioner. 
%
\begin{uppgift}
  Med \verb|\textbf| får man \textbf{fet stil}. Ändra så att ordet
  ''definitioner'' ovan skrivs med fet stil. 
\end{uppgift}
%
En \emph{kryssfråga}         % betonad text
är en fråga till vilken det finns ett mängd \emph{svarsalternativ},
eller bara \emph{svar}, varav minst ett är \emph{korrekt}. 
%
\begin{uppgift}
  Inför följande definition av \emph{fråga}:
%
  \begin{quote}
    En fråga uttrycks med hjälp av en frågesats och är en
    språkhandling avsedd att frambringa information. 
  \end{quote}
%
\end{uppgift}
%
Svar som inte är korrekta kallas \emph{felaktiga}. Ett \emph{prov} är
en mängd kryssfrågor som alla har 
$k \geq 1$        % ger matematisk text, \geq = "greater than or equal"
svarsalternativ varav 
$r \leq k$        % \leq = "less than or equal"
är korrekta.
%
\begin{uppgift}
  Skriv en mening om att det alltid finns minst ett korrekt
  svar. Inkludera en olikhet som relaterar $r$ till 1 (och $k$). 
\end{uppgift}
%
%
% vill man göra manuset luftigare i vertikalled utan att förändra
% manusets betydelse kan man skriva in såna
% här rader med inledande procenttecken
%
%
%
\begin{exempel}    % ett av våre exempel vi tillfört möjligheten att visa
  Om två svar av fem är korrekta på en kryssfråga så är $r = 2$ och
  $k = 5$.
\end{exempel}       % slut på första exemplet
%
\begin{uppgift}
  \label{exuppgift}
  Lägg till ytterligare ett exempel i vilket ett svar av fyra är
  korrekt. 
\end{uppgift}
%
%
% notera att tomrader INTE bör läggas till hur som
% helst, för de betyder "nytt stycke" och gör att det färdiga 
% dokumentets layout förändras
%
%
På varje fråga får högst ett svarsalternativ väljas. Ett korrekt svar
ger 1 poäng medan felaktiga svar och obesvarade frågor ger 0
poäng. Poängen på alla frågorna summeras ihop till provets
\emph{totalpoäng}. Vid chansning, dvs val av svar på måfå, antar vi
att sannolikheten för att välja ett visst svar är likformigt
fördelad. För att resonera omkring vad chansning ger så använder vi
oss av grundläggande sannolikhetslära~\cite{grimaldi}.  
    % med \cite refererar man till en källa, i detta fall 
    % till den enda bok som finns beskriven i slutet av manuset
%
\begin{uppgift}
  I slutet av manuset kan du se hur kommandot \verb|\bibitem| används
  för  att skriva in en referens till en källa -- i detta fall en
  mattebok som i manuset ges nyckelordet \texttt{grimaldi} -- i
  omgivningen \verb|thebibliography|. Här ovan syns hur denna 
  källa sen refereras i löptexten med kommandot
  \verb|\cite{grimaldi}|. I slutet av manuset kan man rada upp
  godtyckligt många källor. Varje källa ska då inledas av
  \verb|\bibitem| och ett unikt nyckelord som får väljas fritt.
%
  \begin{enumerate}
      \item Lägg till följande källa:  
%
        \begin{quote}
          Håkan Jonsson.  \emph{Motivating and Preparing First-Year Students
            in Computer and Engineering Science}.  Proceedings of the
          ASEE/IEEE 43rd Frontiers in Education Conference.  Oklahoma City,
          Oklahoma, USA, Oct 22-26, 2013.
        \end{quote}
%
      \item   Referera sen till den nya källan direkt efter kurskoden
        D0015E på sidan
        \pageref{d0010e}.  %  \pageref ger på vilken sida etiketten finns
        Skriv ett tilde (\verb|~|) som klister mellan kurskoden och
        \verb|\cite|-kommandot med den nya källans nyckelord. 
  \end{enumerate}
%
\end{uppgift}

\section{Antal poäng på en enskild fråga}
\label{sec:question}   
    %  \label definierar en etikett ("sec:question") med 
    %  (i detta fall) avsnittsnumret som värde.
    %  med \ref kan man sen få in värdet i texten
    %  (i detta fall skriver man i så fall  \ref{sec:question} )
    % (man får kalla etiketten annat än sec:...)

\begin{uppgift}
  Ta bort ordet \underline{enskild} från titeln på
  avsnitt~\ref{sec:question}.  
\end{uppgift}
%
Första uppgiften handlar om vad den förväntade poängen på en
kryssfråga är om man bara svarar på måfå. 

\subsection{Sannolikheter}       % ett delavsnitt

Låt $X_i$    % Ett _ inleder ett index; här blir i index till X
vara en stokastisk variabel (även kallad \emph{slumpvariabel}) vars
värde är poängen på fråga $i$. Utfallsrummet $\Omega$  
    %  \Omega ger den grekiska bokstaven "stora O"
    %  Alla grekiska bokstäver finns, både stora och små
som involverar $X_i$ innehåller då de två oberoende händelserna
$\{\text{$X_i$ är 0, $X_i$ är 1}\}$ 
% \{  och \}  ger klammerparenteser, { }, i texten
% \text ger "vanlig" text i matematisk text
som motsvarar 0 respektive 1 poäng på frågan.
%
\begin{uppgift}
  Skriv ett stycke, i vilket du har med två slumpvariabler $X_i$ och
  $X_j$ där $i \not= j$, och förklarar att de är oberoende om deras
  händelser är oberoende. Två händelser är oberoende om  utfallet av
  den ena händelsen inte påverkar utfallet av den andra händelsen.  
\end{uppgift}

För att få poäng ska man välja en av de $r$ korrekta svaren bland de
totalt $k$ svarsalternativ som finns. Enligt den klassiska
sannolikhetsdefinitionen, där sannolikheten är antalet gynnsamma
händelser delat med totala antalet händelser, gäller då att 
%
\begin{equation}             % början av en numrerad ekvation
  \label{eq:2}               % en etikett, nu till ekvationsnumret
  P(X_i = 1) = \frac{r}{k},  % \frac = ett bråk
\end{equation}               % slut på ekvationen
%
där $P$ är en sannolikhetsfunktion. 
%
\begin{uppgift}
  Använd en \texttt{itemize}-omgivning och förklara att det för en
  sannolikhetsfunktion $P$ gäller \underline{dels} att sannolikheten
  $P(H)$ för att en given händelse $H$, av de möjliga händelser som
  kan inträffa, är mindre än eller lika med 1 (vilket går att
  uttrycka matematiskt som en olikhet) \underline{dels} att summan av
  sannolikheterna för alla möjliga händelser är precis 1 (som går att
  uttrycka som en likhet med en summasymbol). 
\end{uppgift}
%
\begin{uppgift}
  Lägg här till den ekvation som du får om du bryter ut $r$ ur
  ekvation~\ref{eq:2}. Skriv nåt i stil med att ''Det betyder t ex
  att...'' och så ekvationen. 
\end{uppgift} 

För komplementhändelsen $X_i = 0$, dvs att svaret är felaktigt, gäller
då att
%
\begin{equation}    % ännu en ekvation
  \label{eq:3}      % ännu en etikett (till denna ekvations nummer)
  \begin{aligned}    
    % en ekvation över flera rader med = under varandra
    % även \begin{align*} ... \end{align*} kan användas och
    % även align utan stjärna får bra men ger automatiskt
    % ekvationsnummer
    P(X_i = 0) &= 1 - P(X_i = 1)  \\
               &= 1 - \frac{r}{k}  \\
               &= \frac{k - r}{k}.
  \end{aligned}
\end{equation}
%
\begin{uppgift} \label{uppgiftA}
  Lägg till en motsvarande uträkning som, rad för rad som i
  uträkningen av ekvation~\ref{eq:3}, istället visar vad
  $P(X_i = 1) \cdot P(X_i=1)$ är uttryckt i $r$ och $k$.
  Alltså, något i stil med:
%
  \begin{equation}   
    \label{ekva}
%
    \begin{aligned}    
      P(X_i = 0) \cdot P(X_i = 1) &= \ldots \\ 
                                  &= \ldots \\
                                  &= \ldots 
    \end{aligned}
  \end{equation}
%
  På sista raden ska du ha nått fram till en produkt $g(r)h(k)$, där
  $g(r)$ är en funktion som beror av $r$ medan $h(k)$ är en funktion
  som beror av $k$. Ett exempel på sådan produkt, som dock \emph{inte}
  är vad du kommer att komma fram till, är
%
  \begin{displaymath}
    4r\frac{k}{k + 1}
  \end{displaymath}
%
  där alltså $g(r)=4r$ och $h(k)=k/(k+1)$. 
\end{uppgift}
%
\begin{uppgift}
  Tänk ut varför produkten $ P(X_i = 0) \cdot P(X_i = 1) $ i
  ekvation~\ref{ekva} alltid är mindre än eller lika med 1. Lägg till
  en mening om det.  
\end{uppgift}
%
Med hjälp av sannolikheterna kan vi beräkna det förväntade
väntevärdet av poängen på fråga $i$. 
%
\begin{uppgift}
  Här är två uppgifter som ankyter till etiketter och referenser till
  etiketter.
%  
  \begin{enumerate}
%
    \item Lägg till en mening som refererar till ekvationen i
      uppgift~\ref{uppgiftA}. Fantisera ihop något om att vi
      \emph{kanske} skulle kunna använda denna ekvation senare i
      rapporten. 
%
    \item Ändra din mening till att avse nästa avsnitt, det om
      väntevärde, genom att lägga till en etikett efter den
      avsnittsrubriken och referera till den.  
  \end{enumerate}
%
\end{uppgift}

\subsection{Väntevärde}

För att beteckna en stokastisk variabels väntevärde använder man
\emph{vänte\-värdes\-operatorn} $E$. 
    % \- (bakåtsnedstreck-bindesstreck) anger var avstavning bör ske,
    % vilket är användbart om ett ord sticker ut i högermarginalen
För en godtycklig stokastisk variabel $A$ är väntevärdet definierat
som
%
\begin{equation*}           % en icke numrerad ekvation
  E[A] = \sum_x xP(A = x),  % \sum = stort summatecken
\end{equation*}
%
där $\sum_x$ betecknar att vi summerar över alla värden som
    % notera storleksskillnaden för \sum_x  i equation resp mellan $ $
slumpvariabeln kan anta. 
%
\begin{uppgift}
  Ändra så att ekvationen blir numrerad.
\end{uppgift}
%
Denna definition ger oss, tillsammans med ekvationerna~\ref{eq:2}
och~\ref{eq:3},   
    %  \ref tar in etikettsvärdena i texten här
att det förväntade antalet poäng är 
%
\begin{align} % flerradsekvation med numrerade rader
  E[X_i] &= 0 \cdot P(X_i = 0) + 1 \cdot P(X_i = 1) \nonumber \\ % ej nummer
         &= P(X_i = 1)                              \nonumber \\ % ej nummer
         &= \boxed{\frac{r}{k}}, \label{eq:oneq}  % denna rad förses med nummer
\end{align}
%
vilket är svaret på uppgift~\ref{item:question}. 
    % \ref{item:question} byts i det färdiga dokumentet ut mot 
    % ut mot numret på den uppgiften

\begin{uppgift}
  Förutom summor kan man uttrycka det mesta matematiskt både vackert
  och funktionellt med \LaTeX, t ex integraler. Med koden
%  
  \verb|$\int_0^x  f(x) \, \text{d}x$| får vi $\int_0^x  f(x) \,
  \text{d}x$ eller  
%
  \begin{displaymath}
    \int_0^x f(x) \, \text{d}x
  \end{displaymath}
%
  om vi sätter uttrycket i en matematikomgivning (\verb|displaymath| i
  detta fall). Storleken på integreringssymbolen anpassas, precis som
  summatecknet tidigare, efter var det ska skrivas ut. Två saker:
%
  \begin{itemize}
    \item Sekvensen \verb|\,| står för ett litet mellanrum när det
      står i matematisk text. Vi behöver tala om för \LaTeX\ att det
      ska vara ett mellanrum för annars skrivs allt ihop som $f(x)
      \text{d} x$ då det ska vara $f(x) \, \text{d} x$. 
%
    \item Sen så används \verb|\text| för att få in vanlig text i
      matematisk text. Vårt d är ju operatorn d och inte variabeln
      $d$. Många struntar dock i denna skillnad och använder $d$ för
      både operator och variabel. Man får göra som man vill så länge
      det inte uppstår tvetydigheter.
  \end{itemize}
%
  Uppgifter:
%
  \begin{enumerate}
    \item Ändra först exemplets övre intervallgräns till 5 och byt sen
      ut $f(x)$ mot $\sin x$. Det senare skrivs \verb|\sin x|.  

      För varje trigonometrisk funktion, t ex sin, finns ett kommando
      i \LaTeX\ som ser till att dess namn skrivs ut rätt i matematisk
      text. Cosinus skrivs t ex \verb|\cos| medan man använder
      \verb|\log| för logaritmer.  
%
    \item Ta bort det inledande bakåtsnedstrecket från \verb|\sin|,
      det vill säga lämna kvar endast \verb|sin|. Observera
      skillnaden. Stoppa tillbaka bakåtsnedstrecket så att $\sin x$
      skrivs ut rätt igen. 
%
    \item Ändra nu gränsen till 10, dvs byt ut \verb|^x| mot
      \verb|^{10}|. 

      Klammerparenteserna behövs för att hålla samman siffrorna. Utan
      klammerparenteser blir exponenten endast den första 1:an. Testa
      med både \verb|^10| och  \verb|^{10}| så ser du denna skillnad. 
%
    \item Ändra istället gränsen till 1000 och byt ut \verb|\sin x| 
      mot polynomet \verb|x^2|. Även denna gräns måste klamras ihop
      som i föregående deluppgift. 

      Tecknet \verb|^|, ibland kallat ''upphöjttecken'', används för
      såväl exponenter som övre gränser på summor och integraler.  
%
    \item Ändra polynomet $x^2$ till ett tredjegradspolynom i
      $x$. Variera termernas koefficienterna.  
%
    \item Ändra slutligen gränsen till \verb|\infty|, oändligheten
      ($\infty$). Här kan man, men behöver inte, använda
      klamrarna. 

      Detta beror på att det inledande bakåtsnedstrecket i
      \verb|\infty| signalerar starten på ett kommando, vilket då
      läses in i sin helhet innan något annat görs. Klamrar fungerar
      på samma sätt. När den vänstra klammerparentesen påträffas så
      signalerar detta starten på något som ska sitta ihop. Detta
      något avslutas i och med att klammerparentesen till höger
      påträffas. När däremot siffor och bokstäver påträffas så
      hanteras de en i taget.
  \end{enumerate}
\end{uppgift}

\begin{exempel}  
    % in med ett exempel, som vi definerat det med newtheorem
    % i början av manuset
  Om två svar av fem är korrekta, och således $r = 2$
  och $k = 5$, är väntevärdet $E[X_i] = 0.4$.
\end{exempel}

\section{Förväntad totalpoäng}
\label{sec:test}

När vi nu vet vilket poäng en enskild fråga förväntas komma att ge kan
vi lösa uppgift~\ref{item:test} och beräkna den förväntade
totalpoängen på ett prov. För detta syfte inför vi ännu en
slumpvariabel 
%
\begin{equation*}
  Y = \sum_{i = 1}^n X_i,
\end{equation*}
%
där $n$ är antalet frågor på provet. 
%
\begin{uppgift}
  Skriv med hjälp av ett summatecken ($\sum$) en ny ekvation som säger
    % \cdots nedan ger tre punkter för "och så vidare" i summor och produkter
  att $Y_{\infty}$ är summan $X_1  + X_2 + \cdots + X_\infty$, poängen på
  ett prov med ett oändligt antal frågor. 
\end{uppgift}
%
Eftersom operatorn $E$ per definition är \emph{linjär}, dvs för alla
$X$ och $Y$ är $E[X + Y] = E[X] + E[Y]$, så är väntevärdet
%
\begin{align*}  % flerradsekvation helt utan numrering
  E[Y] &= E\left(\sum_{i = 1}^n X_i\right) \\
         %  _{i=1} anger startvärde för summan, ^n anger stopvärdet
         %  \left och \right används för att omgärdande paranteser ska
         %  justeras i storlek, i detta fall bli större/högre
       &= \sum_{i = 1}^n E[X_i]
\end{align*}
%
\begin{uppgift}
  Komplettera ekvationen ovan med två förenklingssteg mellan första
  och sista raden enligt följande.
%
  \begin{enumerate}
    \item Lägg först till ett mellansteg där summan inom parantesen i
      högerledet utvecklas medan $E$ står kvar utanför
      paranteserna. Använd \verb|\cdots|. 
%
    \item Lägg till ytterligare ett steg där parantererna tas bort och
      $E$ distribueras in till summans termer.  
\end{enumerate}
%
\end{uppgift}
%
vilket, enligt ekvation~\ref{eq:oneq}, betyder att 
%
\begin{equation*}                              % ekvation, ej numrerad
  E[Y] = \boxed{\frac{rn}{k}} \label{eq:total} % \boxed = rama in
\end{equation*}
%
är svaret på uppgift~\ref{item:test}. % den andra uppgiftens nummer

\begin{exempel}  % ännu ett exempel
  Om vi tänker oss ett konkret fall med $n = 30$ frågor som var och en
  har $k = 5$ svarsalternativ, varav exakt $r = 2$ är korrekta, blir det
  förväntade antalet poäng
%  
  \begin{align*}
    E[Y] &= \frac{2 \cdot 30}{5} \\ % \cdot ger multiplikationstecken
         &= 12
  \end{align*}
%
  om man chansar på alla frågorna.
\end{exempel}

\begin{uppgift}
  Lägg till ett exempel som beräknar den förväntade poängen i en
  situation som den du skrev om i övningsuppgift~\ref{exuppgift}. 
\end{uppgift}

\begin{uppgift}
  Lägg här till ett avslutande avsnitt med rubriken ''Tack'', som
  skapas med kommandot \verb|\section|, i vilket
  du namnger och tackar några lärare, t ex lärare i matematik, du
  tidigare haft för att de hjälpt dig lära dig så mycket att du kunnat
  göra uppgifterna i denna övning. 
\end{uppgift}
%
\begin{thebibliography}{99}    % börja en källförteckning
%
  \bibitem{grimaldi}           % en källa med nyckelord
    Ralph P. Grimaldi, \emph{Discrete and combinatorial mathematics -
    An Applied Introduction}. Addison-Wesley, 2003. 
%
\end{thebibliography}
%
\end{document} % sluta att typsätta, dvs text efter denna rad kommer
               % att ignoreras

