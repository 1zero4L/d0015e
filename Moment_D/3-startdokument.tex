\documentclass[12pt]{article}
\usepackage[utf8]{inputenc}
\usepackage[swedish]{babel}

\title{LaTeX –- ett typsättningssystem}
\author{Håkan Jonsson}
\date{\today}

\begin{document}

\maketitle

\section{Inledning}
I detta dokument skulle kunna ges en kort introduktion av LaTeX. Med den
information som i så fall ges här skulle inlärning och användning av
LaTeX underlättas. Men texten är bara en \emph{fantasiprodukt vars enda syfte
är att användas} för att [\dots] visa på hur LaTeX används.

\section{Nyhetsbrev}
Det finns nyhetsbrev om LaTeX från 1994 och fram till idag. Dessa
lagras i en databas som är sökbar via webben. De flesta går också att
hitta genom att googla. 

\section{Skaffa LaTeX}
LaTeX är fritt tillgängligt och kan installeras på alla moderna
datorer som kör systemen Linux, Mac OS X och Windows. Förutom själva
LaTeX-systemet behöver man också en texteditor, ett program med vilket
man kan upprätta textdokument. LaTeX finns också
tillgängligt online över webben. Då installerar man inte något på sin
egen dator utan använder uteslutande ett LaTeX-system någon annan
installerat och underhåller. 

\subsection{Mac OS X och Linux}
LaTeX finns till Mac OS X och Linux.

\subsection{Windows}
LaTeX finns till Windows.

\section{Hjälp}
Behöver man hjälp så finns det massor av websidor som ägnar stor möda
åt att förklara och lindra besvär. Hela böcker har skrivits om LaTeX.

\end{document}
