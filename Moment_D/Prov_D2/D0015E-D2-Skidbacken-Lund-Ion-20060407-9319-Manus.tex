\documentclass[a4paper,12pt]{article}
\usepackage[swedish]{babel}
\usepackage[utf8]{inputenc}
\usepackage{amsmath, amsthm, amssymb}
% Ändra INTE nästa rad (säger var texten ska typsättas)
\usepackage[a4paper,includeheadfoot,margin=2.54cm]{geometry}
% Ändra INTE nästa rad (som lägger till radnummer till vänster)
\usepackage[left]{lineno}


% Ändra INTE raderna nedan
% Koden är från https://tex.stackexchange.com/questions/43648/
% Den fixar radnumrering av text i närvaro av matematikomgivningar
\newcommand*\patchAmsMathEnvironmentForLineno[1]{%
  \expandafter\let\csname old#1\expandafter\endcsname\csname #1\endcsname
  \expandafter\let\csname oldend#1\expandafter\endcsname\csname end#1\endcsname
  \renewenvironment{#1}%
     {\linenomath\csname old#1\endcsname}%
     {\csname oldend#1\endcsname\endlinenomath}}% 
\newcommand*\patchBothAmsMathEnvironmentsForLineno[1]{%
  \patchAmsMathEnvironmentForLineno{#1}%
  \patchAmsMathEnvironmentForLineno{#1*}}%
\AtBeginDocument{%
\patchBothAmsMathEnvironmentsForLineno{equation}%
\patchBothAmsMathEnvironmentsForLineno{align}%
\patchBothAmsMathEnvironmentsForLineno{flalign}%
\patchBothAmsMathEnvironmentsForLineno{alignat}%
\patchBothAmsMathEnvironmentsForLineno{gather}%
\patchBothAmsMathEnvironmentsForLineno{multline}%
}

% Ändra INTE nästa rad (gör så radnummer skrivs med fet stil)
\renewcommand\linenumberfont{\normalfont\bfseries\small}

\title{Rapport på lösning för uppgift 14}
%
\author{Ion Lund\thanks{email:
        \texttt{ionlun-5@student.ltu.se}}\\  
        ~ \\
        Luleå tekniska universitet \\ 
        971 87 Luleå, Sverige}
%          
\date{22 september 2025}

\begin{document}

\linenumbers % ger radnumrering

\maketitle

\begin{abstract}

  Uppgift 14 är en matematik uppgift från en av Skolverkets prov.
  \cite{Skolverk} Rapporten innehåller detaljerade lösningar för deluppgift 
  A, B, och C. Vi går också genom vad varje fråga vill att vi egentligen ska 
  svara på, och vad våra lösningar egentligen betyder.
\end{abstract}

\section{Introduktion}
\label{sec:introduktion}

  Vi blir medvetna om en skidbacke som har fallhöjden 500 meter. Banprofilen 
  ser du i en bild med höjden $y$ km som är en funktion av sträckan $x$ km.
  \cite{Skolverk} Sambandet mellan $y$ och $x$ ges av att 

  \begin{align}
    y &= 0,5e^{-x^2} \nonumber 
    \\
    0 &\le~x \le 2,5 \nonumber
  \end{align}

\section{Deluppgift A}

\label{sec:uppg1}

  \subsection{Frågan}

    Deluppgift A \cite{Skolverk} vill att man ska lösa för backens lutning där
    $x = 0,8$.
    \\
    Vi ska alltså derivera en sammansatt funktion för att bestämma lutningen 
    \\
    i en viss punkt på backen. I frågan använder vi oss av sambanden:

    \begin{align}
      y &= 0,5e^{-x^2} \nonumber 
      \\
      x &= 0,8 \nonumber
    \end{align}

  \newpage

  \subsection{Lösningen}

    \begin{align}
      \label{eq:1}
      x &= 0,8 \nonumber 
      \\
      y &= 0,5e^{-x²} \nonumber 
      \\
      y^{\prime}(0,8) &= -0,8 \cdot e^{-(0,8²)} 
      \\
      &=-0,42. \nonumber
    \end{align}

  \subsection{Motiveringen}

    Vi bestämde lutningen som $x = 0,8$ och eftersom att lutningen är det 
    samma som derivatan får vi då att $y^{\prime} = -x \cdot e^{-(x²)}$. 
    Därifrån är det bara att byta ut $x$ mot $0,8$ och lösa uppgiften.

\section{Deluppgift B}
\label{sec:uppg2}

    \subsection{Frågan}

      Deluppgift B \cite{Skolverk} vill att vi ska ställa upp en ekvation för 
      bestämning av $x$-värdet.
      I frågan får vi ny relevant information om att ett allmänt sätt att 
      beskriva backar med liknande banprofil som i uppgift A kan ges av 
      funktionen: 

      \begin{displaymath}
        y = 0,5e^{-ax²}
        \quad \text{,} \quad
        0 \le x \le 2,5 
        \text{ där } a \text{ är en positiv konstant.}
      \end{displaymath}

    \subsection{Lösningen}

      \begin{align}
        \label{eq:2}
        %Derivera y.
        y &= 0,5e^{-ax²} \nonumber
        \\
        y^{\prime} &= -ax \cdot e^{-ax²} \nonumber
        \\
        y^{\prime \prime} &= f(x) \cdot g(x) + f^{\prime}(x) \cdot g(x) 
        \nonumber
        \\
        %Förklarar vad funktionerna och dom deriverade funktionerna är. 
        f(x) &= -ax \nonumber 
        \\
        f^{\prime}(x) &= -a \nonumber 
        \\
        g(x) &= e{-ax²} \nonumber 
        \\
        g^{\prime}(x) &= -2ax \cdot e^{-ax²} \nonumber 
        \\
        %Byter ut funktionerna mot variablerna i andraderivatan.
        y^{\prime \prime} &= -a \cdot e^{-ax²} + 2a²x² \cdot e^{\-ax²} 
        \nonumber 
        \\
        &= ae^{-ax²}(2ax²-1) \nonumber 
        \\
        a \cdot e^{-ax²} &> 0 \nonumber 
        \\
        0 &= 2ax²-1 \nonumber 
        \\
        x &= +-\sqrt{\frac{1}{2a}}
      \end{align}

      \newpage
    \subsection{Motiveringen}

      Med hjälp av derivering för $y$-funktionen, så kan vi lösa ut $x$ från 
      potensen.
      \\
      Vi gör en till derivering för att veta funktionens maximi- och 
      minimipunkt. 
      \\
      I den andra deriveringen så får då varje funktion ett värde, och man kan
      då byta ut funktionerna mot variablerna. $a \cdot e^{-ax} > 0$ och vi 
      kan göra om det till en andragradsfunktion och använda en formel för 
      att veta hur vi löser för x.

\section{Deluppgift C}
\label{sec:uppg3}

  \subsection{Frågan}
  
    Deluppgift C \cite{Skolverk} vill att vi ska bestämma $a$ så att backen 
    är brantast för $x = 1,0$.

  \subsection{Lösningen}
  \label{eq:3}
    \begin{align}
      \text{(Tidigare)}
      => a &= \frac{1}{2x²} \nonumber 
      \\
      \text{Om } 
      x = 1 => 
      a &= \frac{1}{2\cdot1²} \nonumber 
      \\
      &= \frac{1}{2} \nonumber 
      \\
      &= 0,5
    \end{align}

  \subsection{Motiveringen}
    Det här är en väldigt simpel lösning. Vi vet redan sen innan att 
    $a = \frac{1}{2x}$, och vi vet också att $x = 1$, och kan då enkelt 
    byta ut variabeln mott dess värde för att få att $a = 0,5$.

\section{Diskussion [och slutsatser]}
\label{sec:disk}
  I den här uppgiften har vi löst tre stycken olika deluppgifter. 
  I deluppgift A \cite{Skolverk} löste vi för backens lutning vid en 
  specifik punkt. Ekvation \ref{eq:1} var en av flera lösningar på 
  problemet, men anledningen till att vi körde med den ekvationen var 
  för att den är väldigt kort och enkel att förstå samt skriva. 
  \\
  \\
  I deluppgit B \cite{Skolverk} ställde vi upp ekvationen nr. \ref{eq:2} för 
  bestämning av $x$-värdet i den punkt där backar med en sån banprofil är 
  brantast. Det var en mer komplicerad uppgift då vi behövde använda oss av 
  andraderivatan, samt att det ser ut som en stor röra av variabler. Men 
  egentligen är den här lösningen så enkelt som det kan bli, och det blir 
  mindre komplicerat när vi skriver definitionerna för funktionerna i 
  andraderivatan. Därifrån kan vi få fram en andragradsfunktion för att lösa 
  vad $x$ blir.
  \\
  \\
  Deluppgift C \cite{Skolverk} är den lättaste uppgiften, då vi får reda på 
  vad $x$ är lika med, samt att tack vare ekvation \ref{eq:2} så vet vi då 
  redan vad $a$ är lika med. Ekvation \ref{eq:3} blir då en enkel uppgift 
  där vi endast behöver skriva in numren för att få vårat svar. 
  \\
  \\
  Hela uppgiften går ut på att vi ska visa våran kunskap inom derivering,
  andraderivering, och funktioner. Det som är utmanande med uppgiften är att
  kunna förstå vad allting ska användas till och hur man gör det, men när 
  man väl förstår teorin bakom problemen och vilka ekvationer och steg som
  kan vara nödvändiga för problemlösningen så är det inga större problem 
  för att klara uppgifterna. 
  %Sammanfatta vad som avhandlats i rapporten, vad du kommit fram till,
  %och sätt det i sitt sammanhang. 
\begin{thebibliography}{99}
%
\bibitem{Skolverk} 
Skolverket.
\textit{Likvärdig bedömning i matematik med stöd av nationella prov}. 
Skolverket, 2004.

\end{thebibliography}
%
\end{document}
