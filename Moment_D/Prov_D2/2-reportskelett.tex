\documentclass[a4paper,12pt]{article}
\usepackage[swedish]{babel}
\usepackage[utf8]{inputenc}
\usepackage{amsmath, amsthm, amssymb}
% Ändra INTE nästa rad (säger var texten ska typsättas)
\usepackage[a4paper,includeheadfoot,margin=2.54cm]{geometry}
% Ändra INTE nästa rad (som lägger till radnummer till vänster)
\usepackage[left]{lineno}


% Ändra INTE raderna nedan
% Koden är från https://tex.stackexchange.com/questions/43648/
% Den fixar radnumrering av text i närvaro av matematikomgivningar
\newcommand*\patchAmsMathEnvironmentForLineno[1]{%
  \expandafter\let\csname old#1\expandafter\endcsname\csname #1\endcsname
  \expandafter\let\csname oldend#1\expandafter\endcsname\csname end#1\endcsname
  \renewenvironment{#1}%
     {\linenomath\csname old#1\endcsname}%
     {\csname oldend#1\endcsname\endlinenomath}}% 
\newcommand*\patchBothAmsMathEnvironmentsForLineno[1]{%
  \patchAmsMathEnvironmentForLineno{#1}%
  \patchAmsMathEnvironmentForLineno{#1*}}%
\AtBeginDocument{%
\patchBothAmsMathEnvironmentsForLineno{equation}%
\patchBothAmsMathEnvironmentsForLineno{align}%
\patchBothAmsMathEnvironmentsForLineno{flalign}%
\patchBothAmsMathEnvironmentsForLineno{alignat}%
\patchBothAmsMathEnvironmentsForLineno{gather}%
\patchBothAmsMathEnvironmentsForLineno{multline}%
}

% Ändra INTE nästa rad (gör så radnummer skrivs med fet stil)
\renewcommand\linenumberfont{\normalfont\bfseries\small}

\title{Rapport på lösning för upgift 14}
%
\author{Ion Lund\thanks{email:
        \texttt{ionlun-5@student.ltu.se}}\\  
        ~ \\
        Luleå tekniska universitet \\ 
        971 87 Luleå, Sverige}
%          
\date{\today}

\begin{document}

\linenumbers % ger radnumrering

\maketitle

\begin{abstract}
  Här skriver du en kort sammanfattning av rapporten som innehåller
  det viktigaste. 
\end{abstract}

\section{Introduktion}
  \label{sec:introduktion}
  Vi blir medvetna om en skidbacke som har fallhöjden 500 meter. Banprofilen ser 
  du i en bild 
  med höjden $y$ km som är en funktion av sträckan $x$ km.
  Sambandet mellan $y$ och $x$ ges av att
  \begin{align}
    \label{eq:1}
    &y = 0,5e^{-x^2} \\
    0 \le~&x \le 2,5
  \end{align}
\section{Deluppgift A}
  \label{sec:uppg1}
  Deluppgift A vill att man ska lösa för backens lutning där $x = 0,8$.\\
  Vi får också veta att ett allmänt sätt att beskriva backar med liknande
  banprofil som i grafen ges av
  \begin{displaymath}
    y = 0,5e^{-ax^2}
    \quad \text{,} \quad 
    0 \le x \le 2,5
    \text{ där $a$ är en positiv konstant.}
  \end{displaymath}
  \newpage
  Eftersom att lutningen är  
  $x = 0,8$ 
  och $y=0,5e^{-x^2}$ 
  vet man då att lutningen är derivatan. Vi får då derivatan:
  \begin{equation}
    y\prime = -x \cdot e^{-(x^4)}
  \end{equation}

\section{Nästa (del-) uppgift}
\label{sec:uppg2}

\section{Och ännu nästa (del-) uppgift...}
\label{sec:uppgN}

\section{Diskussion [och slutsatser]}
\label{sec:disk}

Sammanfatta vad som avhandlats i rapporten, vad du kommit fram till,
och sätt det i sitt sammanhang. 
%
\begin{thebibliography}{99}
%
\bibitem{latexcompanion} 
Michel Goossens, Frank Mittelbach, and Alexander Samarin. 
\textit{The \LaTeX\ Companion}. 
Addison-Wesley, Reading, Massachusetts, 1993.
%
\bibitem{einstein} 
Albert Einstein. 
\textit{Zur Elektrodynamik bewegter K{\"o}rper}. (German) 
[\textit{On the electrodynamics of moving bodies}]. 
Annalen der Physik, 322(10):891–921, 1905.
%
\end{thebibliography}
%
\end{document}
