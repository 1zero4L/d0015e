\documentclass[12pt]{article}
\usepackage[utf8]{inputenc}
\usepackage[swedish]{babel}
\usepackage{a4wide}

\title{Enkel matematik i LaTeX}
\author{Håkan Jonsson}
\date{\today}

\begin{document}

\maketitle

\section{Inledning}
LaTeX kommer bäst till sin rätt då den text man författar innehåller
matematik. Det, samt det faktum att det är fritt tillgängligt och
accepterat som typsättningssyetm över hela värdlen, gör det till ett
kraftfullt verktyg för såväl forskare som ingenjörer. 

I detta dokument visas typsättning av grundläggade matematisk
text. Framställningen är på intet sätt fullständig utan begränsas, för
att bli lätt att överblicka och förstå, medvetet till sånt som är
vanligt förekommande.  

\section{Variabler, konstanter och dollartecken!}

I löptext markeras matematisk text genom att omges av
dollartecken~(\$). Exempel: Identitetsfunktionen $f(x)=x$ som skrivs
\verb|$f(x)=x$|. Det blir snarlikt resultat utan
dollartecken, alltså f(x)=x, men det är \emph{inte} okej att
skriva matematisk text på det sättet. I löptext \emph{ska} matematisk
text märkas ut som just matematik genom att sättas inom
dollartecken. Anledningen är att detta, som du kan se, påverkar
typsättningen, som alltså skiljer sig mellan vanlig text och
matematisk text. 

Observera att i text skrivs bokstäver som de är och om de är betonade
använder man kommandot \verb|\emph|. Men bokstäver som är variabler
ska skrivas inom dollartecken. 

I princip samma sak gäller (numeriska) konstanter. I vanlig text
skriver vi fjorton för det vi i matematisk text skriver som $14$. 

\subsection{Index och sekvenser}

Index på variabler fås med \emph{understrykningstecken}. 

Gruppera ihop index med flera tecken. 

Utelämnade delar i sekvenser markeras med \verb|\ldots|, som gör samma
sak som \verb|\dots| gör i text. 

$x_1, x_2, \ldots, x_{30}$ 

\section{Uttryck}

Variabler och konstanter ihopkombinerade med operatorer och
paranteser.

$a + ( b - c ) / d$

$x^2$

Gruppera ihop exponenter med flera tecken.

\subsection{Multiplikation}

Inte 2 x 3. 

$A \times B$ (''Mängdprodukt'')

$2 \cdot 3$ 

\section{Integraler, summor, produkter och sånt}

$\sum_{i=1}^{k} x_{i} = x_1 + x_2 + \ldots + x_k$

$\int x^2 dx$   (Knepigt!)

$f^{\prime}(x)=\lim_{\varepsilon \to 0} (f(x+\varepsilon)-f(x))/\varepsilon$
\end{document}